\documentclass[a4paper]{article} %
\usepackage{graphicx,amssymb,url} %
\textwidth=12cm \hoffset=-1.2cm %
\textheight=25cm \voffset=-2cm %
\pagestyle{empty} %

\date{} %

\begin{document}


% Please, do not change any of the above lines
% ===========================================
% Type down your paper title
\title{Atomization alternatives in the Russell-Prawitz translation}

% Authors
\author{Gilda Ferreira\\ 
       {\normalsize Universidade Aberta and CMAFcIO (Portugal)} 
       }%

%\author{Author 1, Author 2, Author 3 \\ %
%	{\normalsize Affiliation 1 (Country)} \\ \\
%	Author 4 \\ % If any other author with different Affilation
%	{\normalsize Affiliation 2 (Country)} \\ \\ % Affiliation 2 (if needed)
	% New author \\
	% New affiliation \\
	% Add authors and affiliation as needed
	%{\small \tt{author@university.com}} % Only one corresponding e-mail
%}%


\maketitle

\thispagestyle{empty}


In this presentation, we explore various versions of the Russell-Prawitz translation, which maps intuitionistic propositional calculus to second-order propositional logic. The target system, introduced independently by Jean-Yves Girard and John Reynolds, is known as System F or polymorphic lambda-calculus. We analyze some properties of System F and the translation, with a focus on atomic polymorphism and atomization conversions. Our investigation examines the impact of introducing new atomization conversions to System F \cite{JoseESpiritoSanto} and the more radical option of replacing System F with an atomic polymorphic target system. Specifically, we cover:

\begin{itemize}
	\item[$\bullet$] the original atomic polymorphic variant based on instantiation overflow \cite{Ferreira06,FerreiraFerreira12}.
	\item[$\bullet$] Enhancements such as a variant that improves proof and reduction sequence sizes through an improved form of instantiation overflow \cite{PistoneTranchiniPetrolo}.
	\item[$\bullet$] A variant based on the admissibility of disjunction and absurdity elimination rules, resulting in more compact proves and reduced administrative reductions \cite{JoseESpiritoSanto2019}.
	\item[$\bullet$] An approach that produces an image of intuitionistic propositional calculus truly free from commuting conversions \cite{JoseESpiritoSanto2024}.
\end{itemize}

This exploration includes significant joint work with José Espírito Santo.


\begin{thebibliography}{}
	
\bibitem{JoseESpiritoSanto2019}
J. Esp\'{i}rito Santo, G. Ferreira, A refined interpretation of intuitionistic logic by means of atomic polymorphism, \emph{Studia Logica}, \textbf{108}, pp. 477-507, 2020.\\

\bibitem{JoseESpiritoSanto}
J. Esp\'{i}rito Santo, G. Ferreira, The Russell-Prawitz embedding and the atomization of universal instantiation, \emph{Logic Journal of the IGPL}, \textbf{29(5)}, pp. 823-858, 2021.\\

\bibitem{JoseESpiritoSanto2024}
J. Esp\'{i}rito Santo, G. Ferreira, How to avoid the commuting conversions of IPC, 48 pages. (submitted) https://arxiv.org/pdf/2402.16171.pdf\\

\bibitem{Ferreira06}
        F. Ferreira, Comments on predicative logic, \emph{Journal of Philosophical Logic}, \textbf{35}, pp. 1--8, 2006.\\
  
\bibitem{FerreiraFerreira12}
F. Ferreira, G. Ferreira, Atomic polymorphism, \emph{The Journal of Symbolic Logic}, \textbf{78(1)}, pp. 260-274, 2013.\\

\bibitem{PistoneTranchiniPetrolo}
P. Pistone, L. Tranchini, M. Petrolo, The naturality of natural deduction (II): On atomic polymorphism and generalized propositional connectives, \emph{Studia Logica}, \textbf{110}, pp. 545-592, 2022. 

  \end{thebibliography}
% ===========================================



\end{document}